\documentclass{beamer}

\usetheme{Frankfurt}
\usecolortheme{orchid}

\usepackage{graphicx}
\usepackage{booktabs}
\usepackage{xypic}

\usepackage[english]{babel}
\usepackage[UTF8,noindent]{ctexcap}
\usepackage[bookmarks=true]{hyperref}

\title[MSPS]{Modern Script Processing System}
\subtitle{A Historical and Hierarchical Review}

\author{Zhaoqin Jiang (Kushim)}
\institute{
	Institute of Software, Chinese Academy of Sciences \\ 
	\medskip
	\url{https://github.com/Kushim-Jiang/msps} \\
    \url{mailto:jiangzhaoqin@iscas.ac.cn}
}
\date{\today}

\begin{document}
	
	\begin{frame}
		\titlepage
	\end{frame}
	
	\begin{frame}
		\frametitle{Overview}
		\tableofcontents
	\end{frame}

	% ------------------------
	% Physical Layer
	% ------------------------
	
	\section{Physical Layer}
	
	\begin{frame}
		\frametitle{Physical Text}

		\begin{itemize}
			\item physical text
			\begin{itemize}
				\item \textbf{medium}: physical object as carrier
				\item \textbf{content}: physical object as been carried
			\end{itemize}
			\item ``reading'' process
			\begin{itemize}
				\item \textbf{glottography}: script (writing system)
				\item \textbf{semasiography}: symbol (notation system)
			\end{itemize}
			\item ``writing'' process
			\begin{itemize}
				\item \textbf{writing} (painting): bamboo-silk, papyrus
				\item \textbf{engraving} (carving): oracle-bone, bronze
				\item \textbf{spraying}
				\item \textbf{embossing}: braille
				\item \textbf{debossing}: cuneiform clay tablet
			\end{itemize}
		\end{itemize}
	\end{frame}
	
	\begin{frame}
		\frametitle{Research and Terminology}

		\begin{itemize}
			\item medium
			\begin{itemize}
                \item material classification
                \item production technology: stable material, easy to carry, ...
            \end{itemize}
            \item content
			\begin{itemize}
                \item material classification
                \item production technology: stable material, wide range of color, ...
            \end{itemize}
            \item ``reading'' process
			\begin{itemize}
                \item physiology: eye movement, eye focus, ...
                \item environment: imaging, photography, ...
            \end{itemize}
            \item ``writing'' process
			\begin{itemize}
                \item behavior: calligraphy, penmanship, ...
                \item tool: pen, brush, ...
                \item tool production technology
            \end{itemize}
		\end{itemize}
	\end{frame}
    
	% ------------------------
	% Layout Layer
	% ------------------------
	
    \section{Layout Layer}

	\begin{frame}
		\frametitle{Layout}
		
		\begin{itemize}
			\item physical text reproduction
			\begin{itemize}
                \item demand: different medium with same content
                \item fake reproduction: bind multiple pens
                \item ``macro'' reproduction: \textbf{printing}
                \item \color{gray} ``micro'' reproduction: \textbf{autowriting}
            \end{itemize}
            \item stamp and plate
			\begin{itemize}
				\item \textbf{stamping}: make \textbf{stamp} once, then \textbf{stamp} multiple times
				\item \textbf{printing}: make \textbf{plate} once, then \textbf{print} multiple times
            \end{itemize}
            \item plate and layout
			\begin{itemize}
				\item \textbf{plate}: the generalization of different pages of physical texts
				\item \textbf{layout}: the abstraction of plate, that is, the plate \textbf{without} material information
            \end{itemize}
			\color{gray} \item autowriting
			\begin{itemize}
				\color{gray} \item \textbf{autowriting}: write \textbf{code} once, then \textbf{run} multiple times
				\color{gray} \item \textbf{layout}: the instruction \textbf{without} data structure information
			\end{itemize}
		\end{itemize}
	\end{frame}

	\begin{frame}
		\frametitle{Generalization and Abstraction}

		\begin{columns}[c]
			% left
			\column{.5\textwidth}

			\xymatrix{
				& {layout} \ar[dl]_{print}  \\
				{text}  \ar@<.2em>[r]^{g.} & {plate} \ar[u]_{a.} \ar@<.2em>[l]^{stamp}
			}

			% right
			\column{.5\textwidth}
			
			\xymatrix{
				& {layout} \ar[dl]_{autowrite}  \\
				{text}  \ar@<.2em>[r]^{g.} & {code} \ar[u]_{a.} \ar@<.2em>[l]^{run}
			}
		\end{columns}

	\end{frame}
	
	\begin{frame}
		\frametitle{Research and Terminology}

		\begin{itemize}
			\item layout
			\begin{itemize}
                \item layout element: text block, image, running head, ...
				\item layout metrics: interlinear spacing, head margin, ...
				\item layout specification: hyphenation, indentation, ...
				\item layout style: Villard's Figure, Van de Graaf canon, ...
            \end{itemize}
			\item layout production
			\begin{itemize}
				\item layout conversion: rasterization, coloration, ...
				\item layout design practice: Adobe InDesign, Pages, ...
			\end{itemize}
            \item plate production
			\begin{itemize}
                \item material classification: woodblock, stereotype, lithography, ...
                \item production technology: engraving, casting, ...
                \item middleware: wax, mould, ...
            \end{itemize}
            \item printing process
			\begin{itemize}
                \item GB 9851 (.1 -- .9): \textit{Terminology of Graphic Technology}
            \end{itemize}
		\end{itemize}
	\end{frame}

    \section{Typeface Layer}

	\begin{frame}
		\frametitle{Glyph Image}
		
		\begin{itemize}
			\item plate reproduction
			\begin{itemize}
                \item demand: reuse letter shapes to produce plates
                \item ``macro'' reproduction: \textbf{typesetting}
                \item \color{gray} ``micro'' reproduction: \textbf{typewriting}
            \end{itemize}
            \item type
			\begin{itemize}
				\item \textbf{typesetting} and \textbf{typography}: make \textbf{types}, then \textbf{typeset} multiple times to make \textbf{plate}, then \textbf{print} multiple times
				\item \textbf{font} and \textbf{typeface}: a set of types, and style of them
            \end{itemize}
            \item type and glyph image
			\begin{itemize}
				\item \textbf{type}: the generalization of different shapes in plate
				\item \textbf{glyph image}: the abstraction of type, that is, the type \textbf{without} material information
			\end{itemize}
			\item type as generalization of plate
			\begin{itemize}
				\item \textbf{typesetting}: by appearance
				\item \textbf{phototypesetting}: by size, then by transformation
				\item \textbf{variable font}: by design space
            \end{itemize}
		\end{itemize}
	\end{frame}
	
	\begin{frame}
		\frametitle{Generalization and Abstraction}

		\begin{columns}[c]
			% left
			\column{.6\textwidth}

			\xymatrix{
				& & {glyph~image} \ar[dl]_{design} \\
				& {layout} \ar[dl]_{print} \ar@<.2em>[r]^{g.} & {instance} \ar[u]_{a.} \ar@<.2em>[l]^{combine} \\
				{text} \ar@<.2em>[r]^{g.} & {plate} \ar[u]_{a.} \ar@<.2em>[l]^{stamp} \ar@<.2em>[r]^{g.} & {type} \ar[u]_{a.} \ar@<.2em>[l]^{typeset}
			}

			% right
			\column{.4\textwidth}
			
			\begin{center}
				[typewrite?]
			\end{center}
		\end{columns}

	\end{frame}

	\begin{frame}
		\frametitle{Research and Terminology}

		\begin{itemize}
			\item glyph image
			\begin{itemize}
                \item glyph image data: bitmap, Bézier curve, master, ...
				\item glyph image component: serif, counter, stroke, ascender, ...
				\item glyph image metrics: kerning, optical size, x-height, ...
				\item glyph image style: Garamond, Palatino, Bodoni, Fraktur, ...
			\end{itemize}
			\item glyph image production
			\begin{itemize}
                \item glyph image design (type design) practice: Glyphs, UFO, ...
            \end{itemize}
            \item type production
			\begin{itemize}
                \item material classification: wooden type, metal type, ...
                \item production technology and typesetting technology: Monotype, Linotype, ...
            \end{itemize}
		\end{itemize}
	\end{frame}

    \section{Form Layer}

	\begin{frame}
		\frametitle{Glyph}
		
		\begin{itemize}
			\item font change
			\begin{itemize}
                \item demand: reuse letter shapes to produce plates
                \item ``macro'' reproduction: \textbf{glyph set}
                \item \color{gray} ``micro'' reproduction: \textbf{teletype} (telegram)
            \end{itemize}
			\item glyph image and glyph
			\begin{itemize}
				\item font setting: make glyph sequence, then change different fonts
				\item \textbf{font file}: data file providing design for glyph
				\item \textbf{glyph}: the abstraction of glyph image, that is, the glyph image \textbf{without} shape information
			\end{itemize}
            \item glyph as generalization of glyph image
			\begin{itemize}
				\item \textbf{code character set}: typeface
				\item \textbf{substitution feature}: small caps, old figures, ...
				\item \textbf{variation}: handwriting, punctuation, ...
            \end{itemize}
		\end{itemize}
	\end{frame}
	
	\begin{frame}
		\frametitle{Generalization and Abstraction}

		\begin{columns}[c]
			% left
			\column{.75\textwidth}

			\xymatrix{
				& & & {glyph} \ar[dl]_{design} \\
				& & {glyph~image} \ar[dl]_{design} \ar@<.2em>[r]^{g.} & {glyph~file} \ar[u]_{a.} \ar@<.2em>[l]^{choose} \\
				& {layout} \ar[dl]_{print} \ar@<.2em>[r]^{g.} & {instance} \ar[u]_{a.} \ar@<.2em>[l]^{combine} \ar@<.2em>[r]^{g.} & {layout~file} \ar[u]_{a.} \ar@<.2em>[l]^{choose} \\
				{text} \ar@<.2em>[r]^{g.} & {plate} \ar[u]_{a.} \ar@<.2em>[l]^{stamp} \ar@<.2em>[r]^{g.} & {type} \ar[u]_{a.} \ar@<.2em>[l]^{typeset} \ar@<.2em>[r]^{g.} & {text~file} \ar[u]_{a.} \ar@<.2em>[l]^{choose}
			}

			% right
			\column{.25\textwidth}
			
			\begin{center}
				[teletype?]
			\end{center}
		\end{columns}

	\end{frame}
	
	\begin{frame}
		\frametitle{Research and Terminology}

		\begin{itemize}
			\item graphetics and graphemics (graphematics)
			\begin{itemize}
                \item graphetics: study of writing practice
				\item graphemics: study of writing system
			\end{itemize}
			\item graphemics (graphematics)
			\begin{itemize}
				\item \textbf{graphemics}: ``the study of the writing system from the smallest units to the text'' \textcolor{gray}{(Fuhrhop and Peters, \textit{Einführung in die Phonologie und Graphematik})}
				\item \textbf{grapheme} \& \textbf{graph}: referential conception and analogical conception \textcolor{gray}{(Kohrt, ``The term `grapheme' in the history and theory of linguistics'')}
				\item \textbf{graphotactics}: restrictions on graphemes, such as combination \textcolor{gray}{(McCawley, ``Some graphotactic constraints'')}
				\item \textbf{allography}: graphetic, graphematic (letter a), orthographic (ss \& ß) \textcolor{gray}{(Bunčić, ``A heuristic model for typology'')}
            \end{itemize}
        \end{itemize}
	\end{frame}

    \section{Character Layer}

	\begin{frame}
		\frametitle{Character}
		
		\begin{itemize}
			\item text shaping
			\begin{itemize}
                \item demand: choose different glyphs for the same letter
                \item result: \textbf{character set}
            \end{itemize}
			\item glyph and character
			\begin{itemize}
				\item \textbf{glyph}: letter with actual form (letter a)
				\item \textbf{character}: the abstraction of glyph, that is, the glyph \textbf{without} form information
			\end{itemize}
            \item character as generalization of glyph
			\begin{itemize}
				\item \textbf{cursive joining}: Arabic, Mongolian, ...
				\item \textbf{abstract shape}: CJK Ideographs, Tangut, ...
				\item \textbf{phoneme}: Tai Tham, Tibetan, ...
            \end{itemize}
		\end{itemize}
	\end{frame}
	
	\begin{frame}
		\frametitle{Generalization and Abstraction}

		\xymatrix{
			& & & & {character} \ar[dl]_{shape} \\
			& & & {glyph} \ar[dl]_{design} \ar@<.2em>[r]^{g.} & {variant} \ar[u]_{a.} \ar@<.2em>[l]^{revert} \\
			& & {glyph~image} \ar[dl]_{design} \ar@<.2em>[r]^{g.} & {glyph~file} \ar[u]_{a.} \ar@<.2em>[l]^{choose} \ar@<.2em>[r]^{g.} & {glyph~file} \ar[u]_{a.} \ar@<.2em>[l]^{revert} \\
			& {layout} \ar[dl]_{print} \ar@<.2em>[r]^{g.} & {instance} \ar[u]_{a.} \ar@<.2em>[l]^{combine} \ar@<.2em>[r]^{g.} & {layout~file} \ar[u]_{a.} \ar@<.2em>[l]^{choose} \ar@<.2em>[r]^{g.} & {layout~file} \ar[u]_{a.} \ar@<.2em>[l]^{revert} \\
			{text} \ar@<.2em>[r]^{g.} & {plate} \ar[u]_{a.} \ar@<.2em>[l]^{stamp} \ar@<.2em>[r]^{g.} & {type} \ar[u]_{a.} \ar@<.2em>[l]^{typeset} \ar@<.2em>[r]^{g.} & {text~file} \ar[u]_{a.} \ar@<.2em>[l]^{choose} \ar@<.2em>[r]^{g.} & {text~file} \ar[u]_{a.} \ar@<.2em>[l]^{revert}
		}

	\end{frame}
	
	\begin{frame}
		\frametitle{Research and Terminology}

		\begin{itemize}
			\item character set model
			\begin{itemize}
                \item \textbf{graphemic model}: language weakly-intervened
				\begin{itemize}
					\item natural model
					\item component model
					\item Arabic model
				\end{itemize}
				\item \textbf{phonemic model}: language strongly-intervened
				\begin{itemize}
					\item ligature and conjunct
					\item Brahmi model
					\item phonemic model
				\end{itemize}
			\end{itemize}
			\item text shaping
			\begin{itemize}
				\item \textbf{text shaping}: from character sequence to glyph sequence
				\item \textbf{text representation}: from glyph sequence to character sequence
            \end{itemize}
        \end{itemize}
	\end{frame}

    \section{Hyper-Character Layer}

	\begin{frame}
		\frametitle{Hyper-Character}
		
		\begin{itemize}
			\item character generalization
			\begin{itemize}
                \item synchrony
                \begin{itemize}
					\item example: Simplified Chinese \& Traditional Chinese
					\item Wang Ning: allograph \& alloideograph
					\item Chen Jian: abstract ideograph (?)
				\end{itemize}
                \item diachrony
                \begin{itemize}
					\item evolution of writing systems
				\end{itemize}
            \end{itemize}
		\end{itemize}
	\end{frame}

	\section{Summary}

	\begin{frame}
		\frametitle{Summary}
		
		\begin{itemize}
			\item history and hierarchy
			\begin{itemize}
                \item \textbf{history}: diachrony (not a timeline)
                \item \textbf{hierarchy}: synchrony
                \item hierarchy is historical, history is hierarchical
            \end{itemize}
			\item research and terminology
			\begin{itemize}
                \item \textbf{terminology}: ontological elements
                \item \textbf{research}: ontological structure
                \item terminology and research are historical and hierarchical
            \end{itemize}
		\end{itemize}
	\end{frame}

\end{document} 