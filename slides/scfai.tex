\documentclass{beamer}

\usetheme{Frankfurt}
\usecolortheme{orchid}

\usepackage{graphicx}
\usepackage{booktabs}
\usepackage{xypic}

\usepackage[english]{babel}
\usepackage[UTF8,noindent]{ctexcap}
\usepackage[bookmarks=true]{hyperref}

\title[MSPS]{Modern Script Processing System}
\subtitle{Hierarchy, History and Terminology}

\author{Zhaoqin Jiang (Kushim)}
\institute{
	Institute of Software, Chinese Academy of Sciences \\ 
	\medskip
	\url{https://github.com/Kushim-Jiang/msps} \\
    \url{mailto:jiangzhaoqin@iscas.ac.cn}
}
\date{\today}

\begin{document}
	
	\begin{frame}
		\titlepage
	\end{frame}
	
	\begin{frame}
		\frametitle{Overview}
		\tableofcontents
	\end{frame}

	% ------------------------
	% Physical Layer
	% ------------------------
	
	\section{Physical Layer}
	
	\begin{frame}
		\frametitle{Physical Text}

		\begin{itemize}
			\item physical text
			\begin{itemize}
				\item \textbf{medium}: physical object as carrier
				\item \textbf{content}: physical object as been carried
			\end{itemize}
			\item ``reading'' process
			\begin{itemize}
				\item \textbf{glottography}: script (writing system)
				\item \textbf{semasiography}: symbol (notation system)
			\end{itemize}
			\item ``writing'' process
			\begin{itemize}
				\item \textbf{writing} (painting): bamboo-silk, papyrus
				\item \textbf{engraving} (carving): oracle-bone, bronze
				\item \textbf{spraying}
				\item \textbf{embossing}: braille
				\item \textbf{debossing}: cuneiform clay tablet
			\end{itemize}
		\end{itemize}
	\end{frame}
	
	\begin{frame}
		\frametitle{Research and Terminology}

		\begin{itemize}
			\item medium
			\begin{itemize}
                \item material classification
                \item production technology: stable material, easy to carry, ...
            \end{itemize}
            \item content
			\begin{itemize}
                \item material classification
                \item production technology: stable material, wide range of color, ...
            \end{itemize}
            \item ``reading'' process
			\begin{itemize}
                \item physiology: eye movement, eye focus, ...
                \item environment: imaging, photography, ...
            \end{itemize}
            \item ``writing'' process
			\begin{itemize}
                \item behavior: calligraphy, penmanship, ...
                \item tool: pen, brush, ...
                \item tool production technology
            \end{itemize}
		\end{itemize}
	\end{frame}
    
	% ------------------------
	% Layout Layer
	% ------------------------
	
    \section{Layout Layer}

	\begin{frame}
		\frametitle{Layout}
		
		\begin{itemize}
			\item physical text reproduction
			\begin{itemize}
                \item demand: different medium with same content
                \item fake reproduction: bind multiple pens
                \item ``macro'' reproduction: \textbf{printing}
                \item \color{gray} ``micro'' reproduction: \textbf{autowriting}
            \end{itemize}
            \item stamp and plate
			\begin{itemize}
				\item \textbf{stamping}: make \textbf{stamp} once, then \textbf{stamp} multiple times
				\item \textbf{printing}: make \textbf{plate} once, then \textbf{print} multiple times
            \end{itemize}
            \item plate and layout
			\begin{itemize}
				\item \textbf{plate}: the generalization of different pages of physical texts
				\item \textbf{layout}: the abstraction of plate, that is, the plate \textbf{without} material information
            \end{itemize}
			\color{gray} \item autowriting
			\begin{itemize}
				\color{gray} \item \textbf{autowriting}: write \textbf{code} once, then \textbf{run} multiple times
				\color{gray} \item \textbf{layout}: the instruction \textbf{without} data structure information
			\end{itemize}
		\end{itemize}
	\end{frame}

	\begin{frame}
		\frametitle{Generalization and Abstraction}

		\begin{columns}[c]
			% left
			\column{.45\textwidth}

			\xymatrix{
				& {layout} \ar[dl]_{print}  \\
				{text}  \ar@<.2em>[r]^{g.} & {plate} \ar[u]_{a.} \ar@<.2em>[l]^{stamp}
			}

			% right
			\column{.5\textwidth}
			
			\xymatrix{
				& {layout} \ar[dl]_{autowrite}  \\
				{text}  \ar@<.2em>[r]^{g.} & {code} \ar[u]_{a.} \ar@<.2em>[l]^{run}
			}
		\end{columns}

	\end{frame}
	
	\begin{frame}
		\frametitle{Research and Terminology}

		\begin{itemize}
			\item layout
			\begin{itemize}
                \item layout element: text block, image, running head, ...
				\item layout metrics: interlinear spacing, head margin, ...
				\item layout specification: hyphenation, indentation, ...
				\item layout style: Villard's Figure, Van de Graaf canon, ...
            \end{itemize}
			\item layout production
			\begin{itemize}
				\item layout conversion: rasterization, coloration, ...
				\item layout design practice: Adobe InDesign, Pages, ...
			\end{itemize}
            \item plate production
			\begin{itemize}
                \item material classification: woodblock, sterotype, lithography, ...
                \item production technology: engraving, casting, ...
                \item middleware: wax, mould, ...
            \end{itemize}
            \item printing process
			\begin{itemize}
                \item GB 9851 (.1 -- .9): \textsl{Terminology of Graphic Technology}
            \end{itemize}
		\end{itemize}
	\end{frame}

    \section{Typeface Layer}

	\begin{frame}
		\frametitle{Glyph Image}
		
		\begin{itemize}
			\item plate reproduction
			\begin{itemize}
                \item demand: different plate with same letter shapes
                \item ``macro'' reproduction: \textbf{typesetting}
                \item \color{gray} ``micro'' reproduction: \textbf{typewriting}
            \end{itemize}
            \item type
			\begin{itemize}
				\item \textbf{typesetting}: make \textbf{types}, then \textbf{typeset} multiple times to make \textbf{plate}, then \textbf{print} multiple times
            \end{itemize}
            \item type and glyph image
			\begin{itemize}
				\item \textbf{type}: the generalization of different shapes in plate
				\item \textbf{glyph image}: the abstraction of type, that is, the type \textbf{without} material information
			\end{itemize}
			\item type as generalization of plate
			\begin{itemize}
				\item \textbf{typesetting}: by appearance
				\item \textbf{phototypesetting}: by size, then by transformation
				\item \textbf{variable font}: by design space
            \end{itemize}
		\end{itemize}
	\end{frame}
	
	\begin{frame}
		\frametitle{Generalization and Abstraction}

		\begin{columns}[c]
			% left
			\column{.45\textwidth}

			\xymatrix{
				& & {glyph~image} \ar[dl]_{design} \\
				& {layout} \ar[dl]_{print} \ar@<.2em>[r]^{g.} & {instance} \ar[u]_{a.} \ar@<.2em>[l]^{combine} \\
				{text} \ar@<.2em>[r]^{g.} & {plate} \ar[u]_{a.} \ar@<.2em>[l]^{stamp} \ar@<.2em>[r]^{g.} & {type} \ar[u]_{a.} \ar@<.2em>[l]^{typeset}
			}

			% right
			\column{.5\textwidth}
			
			\begin{center}
				[typewrite?]
			\end{center}
		\end{columns}

	\end{frame}

	\begin{frame}
		\frametitle{Research and Terminology}

		\begin{itemize}
			\item glyph image
			\begin{itemize}
                \item glyph image data: bitmap, Bézier curve, master, ...
				\item glyph image component: serif, counter, stroke, ascender, ...
				\item glyph image metrics: kerning, optical size, x-height, ...
				\item glyph image style: Garamond, Palatino, Bodoni, Fraktur, ...
			\end{itemize}
			\item glyph image production
			\begin{itemize}
                \item glyph image design practice: Glyphs, UFO, ...
            \end{itemize}
            \item type production
			\begin{itemize}
                \item material classification: wooden type, metal type, ...
                \item production technology and typesetting technology: Monotype, Linotype, ...
            \end{itemize}
		\end{itemize}
	\end{frame}

    \section{Form Layer}

	\begin{frame}
	\end{frame}
	
	\begin{frame}
	\end{frame}

    \section{Character Layer}

	\begin{frame}
	\end{frame}
	
	\begin{frame}
	\end{frame}

    \section{Hyper-Character Layer}

	\begin{frame}
	\end{frame}

\end{document} 