\documentclass{beamer}

\usetheme{Copenhagen}
\usecolortheme{orchid}

\usepackage{graphicx}
\usepackage{booktabs}
\usepackage[english]{babel}
\usepackage[UTF8,noindent]{ctexcap}
\usepackage[bookmarks=true]{hyperref}

\title[MSPS]{Modern Script Processing System}

\author{Kushim Jiang}
\institute{
	Institute of Software, Chinese Academy of Sciences \\ 
	\medskip
	\textit{https://github.com/Kushim-Jiang/msps} \\
    \textit{jiangzhaoqin@iscas.ac.cn}
}
\date{\today}

\begin{document}
	
	\begin{frame}
		\titlepage
	\end{frame}
	
	\begin{frame}
		\frametitle{Overview}
		\tableofcontents
	\end{frame}
	
	\section{Physical Layer}
	
    \subsection{Physical Text}
	
	\begin{frame}
		\frametitle{Physical Text}
		\textit{A text is a \textbf{physical object} that can be \textbf{read}, consisting of a set of \textbf{signs} that can be reconstructed by a reader or observer if sufficient interpretants are present, and is evaluated primarily based on its \textbf{content} rather than its physical form or \textbf{medium} of representation.}
	\end{frame}
	
	\subsection{Content of Research}
	
	\begin{frame}
		\frametitle{Content of Research}
		\begin{itemize}
			\item medium production (stable material, easy to carry \ldots)
			\item content production (stable material, wide range of color \ldots)
			\item ``printing'' process
			\begin{itemize}
                \item ``printing'' progress (writing, engraving, printing \ldots)
                \item ``printing'' method (hard material, easy to reproduce \ldots)
            \end{itemize}
            \item reading process (imaging, photography, eye movement \ldots)
		\end{itemize}
	\end{frame}
    
    \section{Typesetting Layer}

    \section{Typeface Layer}

    \section{Form Layer}

    \section{Character Layer}

    \section{Hyper-Character Layer}

\end{document} 